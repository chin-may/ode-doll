\documentclass[11pt]{article}
\usepackage{graphicx}

\author{Group 2 \\Chinmay Bapat (CS10B059)\\ Dhanvin Mehta (CS10B035)}
\title{Simulating interactions using Open Dynamics Engine}
\begin{document}
\maketitle
\section{Introduction}
The Open Dynamics Engine (ODE) is a physics engine. It's two main components
are a rigid body dynamics simulation engine and a collision detection engine.
It has various joint types and collision detection with friction. Each joint
specifies certain constraints. eg. a hinge joint can move about only one
axis of rotation.\\
The world is defined by bodies and certain constants like gravity.
Each iteration of the main loop of ODE engine simulates the physics of these
bodies in the world defined for a fixed time step. By controlling the number
of time steps per frame and the number of frames per second we can control
the sensitivity of the world.

\section{Base code used}
We used ragdoll code from a tutorial to start with. This has the basic
structure and dimensions of the human body but lacks specialized joints and
other constraints. It is just a ragdoll ie it cannot exert any force or 
perform actions. It can only be dropped from a height and falls down as shown
below.

\section{Human Action Simulation}
We have made a realistic humanoid body that can not only stand up,
but can perform various actions. \\
We have implemented the following:\\
\begin{itemize}
    \item Standing stably
    \item Stabilising after being hit by a ball
    \item Walking
    \item Hand Waving
    \item Jumping
    \item Sitting on a block
    \item Kicking a ball
    \item Namaste action
\end{itemize}

These tasks involved putting constraints on the joints for realistic behaviour.
We had to use  \textbf{Finite State Machines} to control the application for
forces for different sections of each action.\\
Apart from this we had to create basic stabilising forces for the body and
pay special attention to smooth transitions between positions. eg. for raising
a leg while kicking, one first transfers his weight onto one leg, and the 
smoothly tilts the other leg backward followed smooth harder motion forward.
Each of these - smooth transitions, shifting weight - requires careful
distribution of forces and torque on the body.
%TODO kick


\section{Human Interaction Simulation}
Interactions between two humans pose additional problems of synchronisation
and additional constraints on the forces one human can apply on the other.
We have implemented the following interactions between two humans:\\
\begin{itemize}
    \item Humans punching each other
    \item Handshaking
\end{itemize}
%TODO pic


\section{Conclusions}
Working with the Open Dynamics Engine made us appreciate the complicated
physics underlying simple human actions. It also helped us realize the 
idea of using Finite State Machines to implement a series of actions.

\end{document}
